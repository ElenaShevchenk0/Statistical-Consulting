\documentclass[12pt,a4paper]{article}
\usepackage[utf8]{inputenc}
\usepackage{amsmath}
\usepackage{amsfonts}
\usepackage{amssymb}
 \usepackage{titling}
 \usepackage{amsthm}
\usepackage{mathrsfs}
\usepackage[export]{adjustbox}
\usepackage{float}

\usepackage{graphicx}
\graphicspath{ {\string~/Desktop/flashcard_/flashcard/usa/consulting/HW6/} }


\title{Flu shots}
\author{Elena Shevchenko}
\date{}
\setlength{\droptitle}{-5cm}

\begin{document}
\maketitle

\subsection*{Data}


A local health clinic sent 
fliers to encourage everyone, but especially older persons at high
risk of complications, to get a 
flu shot in time for protection against an expected 
flu epidemic.
In a pilot follow-up study, 100 clients were randomly selected and asked whether they actually
received a 
flu shot. A client who received a 
flu shot was coded Y = 1, and a client who did
not receive a 
flu shot was coded Y = 0. In addition, data were collected on their age (X1)
and their health awareness. The latter data were combined into a health awareness index
(X2), for which higher values indicate greater awareness. Also included in the data was client
gender, where males were coded X3 = 1 and females were coded X3 = 0.

\subsection*{Analysis}

First data was split into training set and test set. Test set contains 20 observations and the rest 79 points were assigned to the training data. Then logistic regression was fit using training set. Predictors X1 and X2 are significant. Nagelkerke coefficient of determination is 0.41. 

\begin{figure}[H]
\includegraphics[scale=0.8,center]{{log_fit.png}}
\caption{Logistic regression fit}
\end{figure}

ROC curve was plot after applying model to the test data. AUC is 0.89.  Threshold for predicted probabilities is  0.78. Confusion matrix is presented below. The test error rate is 0.9.

\begin{figure}[H]
\includegraphics[scale=0.6,center]{{Rplot1.png}}
\caption{ROC curve}
\end{figure}

\begin{figure}[H]
\includegraphics[center]{{CM.png}}
\caption{Confusion matrix}
\end{figure}

Now LRT and Wald test are conducted to test whether X3 can be removed from the model. Both tests result in choosing the smaller model without X3.  

\begin{figure}[H]
\includegraphics[center]{{LRT.png}}
\caption{LRT for X3}
\end{figure}

\begin{figure}[H]
\includegraphics[center]{{Wald.png}}
\caption{Wald test for X3}
\end{figure}

Results of stepwise model selection using backward method show that the best model is the model with only two predictors X1 and X2.

\begin{figure}[H]
\includegraphics[scale=0.8,center]{{stepwise.png}}
\caption{Model selection}
\end{figure}


\subsection*{Prediction}

Now prediction is made to determine the probability that male client aged 55 with a health awareness index at the average health awareness score will receive a  flu shot using the final 2-factor model. The probability is 0.054, which is much lower than the threshold, meaning that this person will unlikely receive a flu shot.



\end{document}



